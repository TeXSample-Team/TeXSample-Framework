\documentclass{article}

\usepackage[dvipdfm]{graphicx}
\usepackage{curves}
\usepackage{amssymb}
\usepackage{amsfonts}
\usepackage{amsmath}
\usepackage{makeidx}
\usepackage[utf8]{inputenc}
\usepackage[russian]{babel}
\usepackage{ucs}
\usepackage{texsample}
\usepackage{wallpaper}
\usepackage{rotating}
\usepackage[bookmarksnumbered,colorlinks,plainpages,backref,unicode]{hyperref}

\hypersetup{pdfpagemode=FullScreen}

\newsavebox{\boxA}
\newsavebox{\boxB}
\newsavebox{\boxC}
\newsavebox{\boxD}
\newsavebox{\boxE}
\newsavebox{\boxF}
\newsavebox{\boxG}
\newsavebox{\boxH}
\newsavebox{\boxI}
\newsavebox{\boxJ}
\newsavebox{\boxK}
\newsavebox{\boxL}
\newsavebox{\boxM}
\newsavebox{\boxN}
\newsavebox{\boxO}
\newsavebox{\boxP}
\newsavebox{\boxQ}
\newsavebox{\boxR}
\newsavebox{\boxS}
\newsavebox{\boxT}
\newsavebox{\boxU}
\newsavebox{\boxV}
\newsavebox{\boxW}
\newsavebox{\boxX}
\newsavebox{\boxY}
\newsavebox{\boxZ}

\makeindex

 \paperwidth=190mm
 \paperheight=130mm
 \textwidth=180mm
 \textheight=120mm
 \evensidemargin=-20mm
 \oddsidemargin=-20mm
 \topmargin=-10mm
 \headheight=-10mm
 \headsep=0mm

 \emergencystretch=16pt


\begin{document}

\newcommand{\mylabel}[1]{\label{#1} \hypertarget{HT#1}{}}

\def\ifundefined#1{\expandafter\ifx\csname#1\endcsname\relax}

\newcounter{MakExecisePdf}
\setcounter{MakExecisePdf}{0}
\renewcommand{\theMakExecisePdf}{{\Roman{section}.}\arabic{MakExecisePdf}}
\newcounter{MakExecisePdfPrint}
\setcounter{MakExecisePdfPrint}{0}
 \def\MakExecisePdf#1#2{\newpage\global\refstepcounter{MakExecisePdf}
   \expandafter\newcounter{MakExecise\roman{MakExecisePdf}Pdf}
   \mbox{\textbf{Задача~{\Roman{section}.}\arabic{MakExecisePdf}.}}
   \addcontentsline{toc}{section}{Задача~{\Roman{section}.}\arabic{MakExecisePdf}}
   {\large (\hyperlink{HTMakExecisePdf\arabic{MakExecisePdf}answ}{Ответ приведен на
   стр.\pageref{MakExecisePdf\arabic{MakExecisePdf}answ}}.)}
   \mylabel{MakExecisePdf\arabic{MakExecisePdf}}
   \expandafter\def\csname
   MakExecise\roman{MakExecisePdf}text\endcsname{#1}
   \expandafter\def\csname
   MakExecise\roman{MakExecisePdf}answ\endcsname{#2}
   \csname MakExecise\roman{MakExecisePdf}text\endcsname
   \sloppy\par}

 \def\MakExPrinAnsw{\setcounter{MakExecisePdfPrint}{0}
  \loop
   \ifnum\arabic{MakExecisePdfPrint}<\arabic{MakExecisePdf}
   \newpage
   \refstepcounter{MakExecisePdfPrint}
   \noindent\centerline{\mbox{\Huge\textbf{Решение задачи~\arabic{MakExecisePdfPrint}.}}}\par
   \hyperlink{HTMakExecisePdf\arabic{MakExecisePdfPrint}}
   {\mbox{\textbf{Задача~\arabic{MakExecisePdfPrint}.}}}\quad
   \csname MakExecise\roman{MakExecisePdfPrint}text\endcsname
   \newpage
   \hyperlink{HTMakExecisePdf\arabic{MakExecisePdfPrint}}
   {\mbox{\textbf{Задача~\arabic{MakExecisePdfPrint}.}}}~%
   \csname MakExecise\roman{MakExecisePdfPrint}text\endcsname
   \mylabel{MakExecisePdf\arabic{MakExecisePdfPrint}answ}
   \sloppy\par
   {\bf Ответ.}~\csname MakExecise\roman{MakExecisePdfPrint}answ\endcsname
  \repeat}

\def\MakeExNewpage{\newpage
   {%\LARGE
   \hyperlink{HTMakExecisePdf\arabic{MakExecisePdfPrint}}
   {\mbox{\textbf{Задача~\arabic{MakExecisePdfPrint}.}}}~%
   \csname MakExecise\roman{MakExecisePdfPrint}text\endcsname\newline}
\rule{\parindent}{0pt}\textbf{Ответ.}~}

\def\ClearExNewpage{\newpage
   {%\LARGE
   \hyperlink{HTMakExecisePdf\arabic{MakExecisePdfPrint}}
   {\mbox{\textbf{Задача~\arabic{MakExecisePdfPrint}.}}}~%
   \newline}}


\newcounter{MakRemarkPdf}
\setcounter{MakRemarkPdf}{0}
\renewcommand{\theMakRemarkPdf}{{\Roman{section}.}\arabic{MakRemarkPdf}}
\newcounter{MakRemarkPdfPrint}
\setcounter{MakRemarkPdfPrint}{0}
 \def\MakRemarkPdf#1#2{\newpage\global\refstepcounter{MakRemarkPdf}
   \expandafter\newcounter{MakRemark\roman{MakRemarkPdf}Pdf}
   \textbf{Замечание~{\Roman{section}.}\arabic{MakRemarkPdf}.}
   \addcontentsline{toc}{section}{Замечание~{\Roman{section}.}\arabic{MakRemarkPdf}}
   {\large (\hyperlink{HTMakRemarkPdf\arabic{MakRemarkPdf}answ}{Ответ приведен на
   стр.\pageref{MakRemarkPdf\arabic{MakRemarkPdf}answ}}.)}
   \mylabel{MakRemarkPdf\arabic{MakRemarkPdf}}
   \expandafter\def\csname
   MakRemark\roman{MakRemarkPdf}text\endcsname{#1}
   \expandafter\def\csname
   MakRemark\roman{MakRemarkPdf}answ\endcsname{#2}
   \csname MakRemark\roman{MakRemarkPdf}text\endcsname
   \sloppy\par}

 \def\MakRemPrinAnsw{\setcounter{MakRemarkPdfPrint}{0}
  \loop
   \ifnum\arabic{MakRemarkPdfPrint}<\arabic{MakRemarkPdf}
   \newpage
   \refstepcounter{MakRemarkPdfPrint}
   \textbf{Замечание~{\Roman{section}.}\arabic{MakRemarkPdfPrint}.}~%
   \csname MakRemark\roman{MakRemarkPdfPrint}text\endcsname
   \mylabel{MakRemarkPdf\arabic{MakRemarkPdfPrint}answ}
   \sloppy\par
   {\bf Ответ.}~%
   \csname MakRemark\roman{MakRemarkPdfPrint}answ\endcsname
  \repeat}







\def\TwoBlock#1#2#3#4#5{{\unitlength=\textwidth%
     {\makebox(#1,#3)[lt]{\parbox{#1\textwidth}{#4}}}~~%
     {\makebox(#2,#3)[rt]{\parbox{#2\textwidth}{#5}}}}}

\def\fTwoBlock#1#2#3#4#5{{\unitlength=\textwidth%
     \framebox{\makebox(#1,#3)[lt]{\parbox{#1\textwidth}{#4}}}~~%
     \framebox{\makebox(#2,#3)[rt]{\parbox{#2\textwidth}{#5}}}}}

\newcounter{TabCountGlb}
\newcounter{RisCountGlb}
\setcounter{TabCountGlb}{0} \setcounter{RisCountGlb}{0}


\newcounter{zaplatka}
\setcounter{zaplatka}{0}
\newcounter{zaplatPrint}
\setcounter{zaplatPrint}{0}
\newcounter{setka}
\def\setka{%
     \multiput(-170,-110)(5,0){69}{%
        \multiput(0,0)(0,1){220}{\circle*{0.3}}}
     \multiput(-170,-100)(0,5){45}{%
        \multiput(0,0)(1,0){340}{\circle*{0.3}}}
      \multiput(-163,00.1)(10,0){34}{{\thicklines\line( 1, 0){6}}}
      \multiput(-163,-0.1)(10,0){34}{{\thicklines\line( 1, 0){6}}}
       \multiput(00.1,-113)(0,10){22}{{\thicklines\line( 0, 1){6}}}
       \multiput(-0.1,-113)(0,10){22}{{\thicklines\line( 0, 1){6}}}
     \setcounter{setka}{0}
     \multiput(10,-1)(10,0){17}{\addtocounter{setka}{10}\makebox(0,0)[ct]{{\large\arabic{setka}}}}
     \setcounter{setka}{0}
     \multiput(-1,10)(0,10){11}{\addtocounter{setka}{10}\makebox(0,0)[rc]{{\large\arabic{setka}}}}
     \setcounter{setka}{0}
     \multiput(-10,-1)(-10,0){9}{\addtocounter{setka}{10}\makebox(0,0)[ct]{{\large-\arabic{setka}}}}
     \setcounter{setka}{0}
     \multiput(-1,-10)(0,-10){09}{\addtocounter{setka}{10}\makebox(0,0)[rc]{{\large-\arabic{setka}}}}
     }
\def\zaplatka#1{\par\medskip
                \refstepcounter{zaplatka}\mylabel{Zapl\arabic{zaplatka}}
                \centerline{\rule{0.3\textwidth}{0.4pt} {\LARGE\bf
                            Заплатка~\arabic{zaplatka}}
                            \rule{0.3\textwidth}{0.4pt}}
                \par{\sf #1}
                \expandafter\def\csname ZaplaText\roman{zaplatka}\endcsname{#1}
                \par\medskip\centerline{\rule{0.8\textwidth}{0.4pt}}
                \par\medskip}


\newcounter{MyCommentPdfMake}
\setcounter{MyCommentPdfMake}{0}
\newcounter{MyCommentPdfPrint}
\setcounter{MyCommentPdfPrint}{0}
 \def\MyCommentPdf#1{\par\medskip
                \refstepcounter{MyCommentPdfMake}\mylabel{MyComm\arabic{MyCommentPdfMake}}
                \textsf{\textbf{Ссылка~\arabic{MyCommentPdfMake}.}
                  Если возникли вопросы, см.
                  \hyperlink{HTMyCommentPdfPrint\arabic{MyCommentPdfMake}}{комментарий на
                  стр.~\pageref{MyCommentPdfPrint\arabic{MyCommentPdfMake}}}.}
                  \global\expandafter\def\csname MyCommText\roman{MyCommentPdfMake}\endcsname{#1}
                \par\medskip}

 \def\MyCommentPdfPrint{%
      \setcounter{MyCommentPdfPrint}{0}
       \ifnum\value{MyCommentPdfMake}>0
       \loop
       \ifnum\value{MyCommentPdfPrint}<\value{MyCommentPdfMake}
       \refstepcounter{MyCommentPdfPrint}
       \mylabel{MyCommentPdfPrint\arabic{MyCommentPdfPrint}}
       {\Large
       \hyperlink{HTMyComm\arabic{MyCommentPdfPrint}}
       {\textbf{Комментарий к ссылке~\ref{MyComm\arabic{MyCommentPdfPrint}} на
      стр.~\pageref{MyComm\arabic{MyCommentPdfPrint}}.}}
       \expandafter\csname MyCommText\roman{MyCommentPdfPrint}\endcsname
       \newpage}
       \repeat
      \fi}



\newcommand{\bookname}{Векторная алгебра и аналитическая геометрия}
\newcommand{\booksubname}[1]{Для студентов заочного факультета всех
                             специальностей}
\renewcommand{\figurename}{Рис.}
\renewcommand{\contentsname}{Оглавление}
\renewcommand{\chaptername}{Глава}
\renewcommand{\indexname}{Предметный указатель}
\renewcommand{\tablename}{Таблица}
\renewcommand{\refname}{Список литературы}
\renewcommand{\bibname}{Список литературы}
\renewcommand{\indexname}{Предметный указатель}
\newtheorem{defnt}{\noindent \bf Определение}%[section]
\newtheorem{thm}{\noindent \bf Теорема}%[section]
\newtheorem{lmm}{\noindent \bf Лемма}%[section]
\newtheorem{prop}{\noindent \bf Утверждение}[section]
\newtheorem{prim}{\noindent \bf Пример}%[section]
\newtheorem{zad}{\noindent \bf Задача}%[section]
\newtheorem{quest}{\noindent \bf Вопрос}%[section]
\newtheorem{cons}{\noindent \bf Следствие}%[section]
\newtheorem{zam}{\noindent \bf Замечание}%[section]
\newtheorem{sogl}{\noindent \bf Соглашение}%[section]
\newtheorem{aim}{\noindent \bf Цель}%[section]

\newcommand{\defntName}[1]{\medskip\noindent\mbox{\bf Определение~#1.}}%[section]
\newcommand{\thmName}[1]{\medskip\noindent\mbox{\bf Теорема~#1.}}%[section]
\newcommand{\lmmName}[1]{\medskip\noindent\mbox{\bf Лемма~#1.}}%[section]
\newcommand{\propName}[1]{\medskip\noindent\mbox{\bf Утверждение~#1.}}%[section]
\newcommand{\primName}[1]{\medskip\noindent\mbox{\bf Пример~#1.}}%[section]
\newcommand{\zadName}[1]{\medskip\noindent\mbox{\bf Задача~#1.}}%[section]
\newcommand{\questName}[1]{\medskip\noindent\mbox{\bf Вопрос~#1.}}%[section]
\newcommand{\consName}[1]{\medskip\noindent\mbox{\bf Следствие~#1.}}%[section]
\newcommand{\zamName}[1]{\medskip\noindent\mbox{\bf Замечание~#1.}}%[section]
\newcommand{\soglName}[1]{\medskip\noindent\mbox{\bf Соглашение~#1.}}%[section]
\newcommand{\aimName}[1]{\medskip\noindent\mbox{\bf Цель~#1.}}%[section]

\newcommand{\Arg}{\mathop{\rm Arg}\nolimits}
\newcommand{\tr}{\mathop{\rm tr}\nolimits}
\newcommand{\Rg}{\mathop{\rm Rg}\nolimits}
\newcommand{\Ker}{\mathop{\rm Ker}\nolimits}
\newcommand{\spec}{\mathop{\rm spec}\nolimits}
\newcommand{\diag}{\mathop{\rm diag}\nolimits}
\newcommand{\vectrup}[1]{\vectr{#1}\hspace{-0.5em}\rule{0pt}{2.5
                         ex}'\hspace{0.2em}\rule{0pt}{1ex}}
\newcommand{\nablavec}{\overrightarrow{\bigtriangledown}}
\newcommand{\nablacoord}[1]{\bigtriangledown_{#1}}

\newcommand{\vectr}[1]{\overrightarrow{\bf #1}}
\newcommand{\prc}{\mathop{\mbox{\rm пр}}\nolimits}
\newcommand{\diez}{\mathop{\#}\limits}
\newcommand{\supp}{\mathop{\rm supp}\nolimits}
\newcommand{\Irr}{\mathop{\rm Irr}\nolimits}
\newcommand{\bigref}[1]{~\ref{#1}, стр.\pageref{#1}}
\newcommand{\mathref}[1]{~(\ref{#1}), стр.\pageref{#1}}
\newcounter{bookyear}
\setcounter{bookyear}{2012}

\def\izdat#1{\ifnum#1<2 Редакционно-издательский отдел ГОУ ВПО УГТУ-УПИ\fi
             \ifnum#1=2 Редакционно-издательский отдел ГОУ ВПО УГТУ-УПИ\fi }
\def\adressIzdat{620002, Екатеринбург, ул.~Мира, 19}
\def\ISBN{ISBN ?-???-??????-?}

\newcommand{\PdfSection}[9]{\newpage\noindent
\ifnum#2=0\refstepcounter{section}\fi{\huge\bf\thesection.~#1}%
\ifnum#2=0\addcontentsline{toc}{section}{\thesection.~#1}\fi}

\newcommand{\PdfSubSection}[9]{\newpage\noindent
\ifnum#2=0\refstepcounter{subsection}\fi{\huge\bf\thesubsection.~#1}%
\ifnum#2=0\addcontentsline{toc}{subsection}{\thesubsection.~#1}\fi}

\newcommand{\PdfSubSubSection}[9]{\newpage\noindent
\ifnum#2=0\refstepcounter{subsubsection}\fi{\huge\bf\thesubsubsection.~#1}%
\ifnum#2=0\addcontentsline{toc}{subsubsection}{\thesubsubsection.~#1}\fi}



\newcommand{\PrimCom}[9]{%
    %#1 --- название примера
    %#2 --- метка
    %#3 --- текст примера
   \expandafter\newcommand\csname
   ComTextPrim#2\endcsname[6]{
    \ifundefined{ComPrim#2}
     \newpage\noindent
     \expandafter\newcommand\csname ComPrim#2\endcsname[6]{#3}
     \begin{prim}\expandafter\mylabel{Prim#2}
      \addcontentsline{toc}{section}{Пример~\arabic{prim} #1}
      \csname ComPrim#2\endcsname{#4}{#5}{#6}{#7}{#8}{#9}
     \end{prim}
    \else
      \newpage\noindent
     \expandafter\renewcommand\csname ComPrim#2\endcsname[6]{#3}
     \addtocounter{prim}{-2}\refstepcounter{prim}
     \begin{prim}
      \csname ComPrim#2\endcsname{#4}{#5}{#6}{#7}{#8}{#9}
     \end{prim}
    \fi}
  }


\newcommand{\ThmEnvCom}[9]{%
    %#1 --- тип окружения: thm, lmm, defnt, zam
    %#2 --- метка
    %#3 --- формулировка теоремы, определения и др.
   \expandafter\newcommand\csname ComText#1#2\endcsname[6]{
    \ifundefined{Com#1#2}
     \expandafter\newcommand\csname Com#1#2\endcsname[6]{#3}
     \csname #1Name\endcsname{\ref{#1#2}}\refstepcounter{#1}\expandafter\mylabel{#1#2}%
      \textit{\csname Com#1#2\endcsname{\#1}{\#2}{\#3}{\#4}{\#5}{\#6}}
    \else
     \expandafter\renewcommand\csname Com#1#2\endcsname[6]{#3}
     \csname #1Name\endcsname{\ref{#1#2}}
      \textit{\csname Com#1#2\endcsname{\#1}{\#2}{\#3}{\#4}{\#5}{\#6}}
    \fi}
  }


\newlength{\PictureWidthAw}
\newlength{\ParagraphWidthAw}
\newcounter{PictureWidthAn}
\newcounter{ParagraphWidthAn}

\newcommand{\PictText}[9]{\noindent{\unitlength=1mm%
\setcounter{PictureWidthAn}{#1}%
\setcounter{ParagraphWidthAn}{180}%
\addtocounter{ParagraphWidthAn}{-\arabic{PictureWidthAn}}%
\addtocounter{ParagraphWidthAn}{-7}%
\setlength{\PictureWidthAw}{\arabic{PictureWidthAn}mm}%
\setlength{\ParagraphWidthAw}{\arabic{ParagraphWidthAn}mm}%
{\makebox(\arabic{PictureWidthAn},#2)[lt]{%
 \begin{picture}(\arabic{PictureWidthAn},#2)(-#3,-#4)
  #5
 \end{picture}}}\quad
{\makebox(\arabic{ParagraphWidthAn},#2)[lt]{\parbox{\ParagraphWidthAw}{#6}}}}}

\newcommand{\TextPict}[9]{\noindent{\unitlength=1mm%
\setcounter{PictureWidthAn}{#1}%
\setcounter{ParagraphWidthAn}{180}%
\addtocounter{ParagraphWidthAn}{-\arabic{PictureWidthAn}}%
\addtocounter{ParagraphWidthAn}{-7}%
\setlength{\PictureWidthAw}{\arabic{PictureWidthAn}mm}%
\setlength{\ParagraphWidthAw}{\arabic{ParagraphWidthAn}mm}%
\mbox{\makebox(\arabic{ParagraphWidthAn},#2)[lt]{\parbox{\ParagraphWidthAw}{#6}}\quad
\makebox(\arabic{PictureWidthAn},#2)[lt]{%
 \begin{picture}(\arabic{PictureWidthAn},#2)(-#3,-#4)
  #5
 \end{picture}}}}}


\newcommand{\diam}{\mathop{\mathrm{diam}}}
\newcommand{\grad}{\mathop{\mathrm{grad}}}
\newcommand{\rot}{\mathop{\mathrm{rot}}}
\newcommand{\divrg}{\mathop{\mathrm{div}}}


\LARGE \tolerance=6000